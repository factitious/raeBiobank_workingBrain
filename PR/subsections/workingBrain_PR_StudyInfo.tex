% Options for packages loaded elsewhere
\PassOptionsToPackage{unicode}{hyperref}
\PassOptionsToPackage{hyphens}{url}
%
\documentclass[
]{article}
\usepackage{lmodern}
\usepackage{amssymb,amsmath}
\usepackage{ifxetex,ifluatex}
\ifnum 0\ifxetex 1\fi\ifluatex 1\fi=0 % if pdftex
  \usepackage[T1]{fontenc}
  \usepackage[utf8]{inputenc}
  \usepackage{textcomp} % provide euro and other symbols
\else % if luatex or xetex
  \usepackage{unicode-math}
  \defaultfontfeatures{Scale=MatchLowercase}
  \defaultfontfeatures[\rmfamily]{Ligatures=TeX,Scale=1}
\fi
% Use upquote if available, for straight quotes in verbatim environments
\IfFileExists{upquote.sty}{\usepackage{upquote}}{}
\IfFileExists{microtype.sty}{% use microtype if available
  \usepackage[]{microtype}
  \UseMicrotypeSet[protrusion]{basicmath} % disable protrusion for tt fonts
}{}
\makeatletter
\@ifundefined{KOMAClassName}{% if non-KOMA class
  \IfFileExists{parskip.sty}{%
    \usepackage{parskip}
  }{% else
    \setlength{\parindent}{0pt}
    \setlength{\parskip}{6pt plus 2pt minus 1pt}}
}{% if KOMA class
  \KOMAoptions{parskip=half}}
\makeatother
\usepackage{xcolor}
\IfFileExists{xurl.sty}{\usepackage{xurl}}{} % add URL line breaks if available
\IfFileExists{bookmark.sty}{\usepackage{bookmark}}{\usepackage{hyperref}}
\hypersetup{
  hidelinks,
  pdfcreator={LaTeX via pandoc}}
\urlstyle{same} % disable monospaced font for URLs
\usepackage[margin=1in]{geometry}
\usepackage{graphicx,grffile}
\makeatletter
\def\maxwidth{\ifdim\Gin@nat@width>\linewidth\linewidth\else\Gin@nat@width\fi}
\def\maxheight{\ifdim\Gin@nat@height>\textheight\textheight\else\Gin@nat@height\fi}
\makeatother
% Scale images if necessary, so that they will not overflow the page
% margins by default, and it is still possible to overwrite the defaults
% using explicit options in \includegraphics[width, height, ...]{}
\setkeys{Gin}{width=\maxwidth,height=\maxheight,keepaspectratio}
% Set default figure placement to htbp
\makeatletter
\def\fps@figure{htbp}
\makeatother
\setlength{\emergencystretch}{3em} % prevent overfull lines
\providecommand{\tightlist}{%
  \setlength{\itemsep}{0pt}\setlength{\parskip}{0pt}}
\setcounter{secnumdepth}{-\maxdimen} % remove section numbering

\author{}
\date{\vspace{-2.5em}}

\begin{document}

\hypertarget{background}{%
\subsection{Background}\label{background}}

Work takes up a huge chunk of our adult lives: the average Briton works
approximately \(42\) hours per week {[}@TUC1{]}, with an additional
\textasciitilde4.9 hours spent on commuting {[}@TUC2{]}, and an estimate
of \textasciitilde10.1 hours in unpaid overtime {[}@TotallyMoney{]}.
These numbers have been growing in the past 30 years {[}@TUC1; @TUC2;
@TotallyMoney{]} without benefits to productivity. Importantly, a
growing body of evidence suggests a strong negative impact on our health
and wellbeing. Long working hours are associated with a higher risk of
cardiovascular disease {[}@Bannai2014{]}, higher incidence of depressive
{[}@Kim2016{]}, and anxiety symptoms {[}@Bannai2014{]}, deficient
cognitive function {[}@Kajitani2018{]}, and adverse physiological
changes {[}@VanderHulst2003{]}. Moreover, interventional studies show
that a reduction in working hours benefits both health and productivity
{[}@Akerstedt2001; @Barck-Holst2017{]}. However, we do not yet
understand the neurobiological implications of our modern, increasingly
intense, working patterns. Three reasons motivate the need for such an
understanding:

\begin{itemize}
\item
  The brain acts as an interface between the body and the environment,
  therefore, it is key for grasping the mechanism through which
  occupational factors are affecting our health and wellbeing.
\item
  Without it we cannot ascertain the true short-term impact of working
  patterns on our cognitive function and physiological health, let alone
  the long-term, potentially irreversible, effects on our mental health
  and wellbeing.
\item
  Scientific evidence is needed to inform public policy and industry
  standards surrounding healthy work patterns.
\end{itemize}

\newpage

\hypertarget{aim-and-objectives}{%
\subsection{Aim and objectives}\label{aim-and-objectives}}

This project aims to characterize the neurophysiological processes
through which work patterns and other occupational factors affect our
health and wellbeing, with the following objectives:

\begin{itemize}
\tightlist
\item
  Identify occupational factors that have a meaningful impact on
  neuronal function and describe the mechanism of impact.
\item
  Assess how physiological inflammatory responses are altered by
  occupational factors.
\item
  Determine how the identified neuronal and inflammatory markers jointly
  affect our physical and mental health.
\end{itemize}

Progress against these objectives will help develop a holistic insight
into why our wellbeing is affected by modern work patterns and other
occupational factors.

\newpage

\hypertarget{rationale}{%
\subsection{Rationale}\label{rationale}}

The impact of working patterns on our neurophysiology is still not
comprehensively understood; therefore our investigative plan does not
build solely and directly on prior work in this specific arena. However,
we have identified inadequate sleep as one principal means through which
the influence of occupational factors on wellbeing is likely to
manifest. First, relative to all other activities, work is the primary
waking activity exchanged for sleep {[}@Basner2014{]}. Second, it is
increasingly common for workers to accumulate sleep debt throughout the
working week and attempt to catch-up on the weekend, a countermeasure
that has been shown to often be ineffective in combating the deleterious
effects of weekday sleep debt {[}@Basner2018; @Simpson2016;
@Reinhardt2016; @Buxton2010{]}. Finally, working longer hours is
associated with significantly reduced sleep duration and quality
{[}@Marucci-Wellman2016, @VanderHulst2003{]}. Therefore, we will use the
known neuronal and physiological mechanisms of sleep, and in particular
sleep restriction, to help guide the incipient stage of our
investigation.

A multitude of bodily systems react to and interact with sleep-loss, a
key set being the body's inflammatory response, and in particular
increased expression of proinflammatory cytokines. Sleep restriction
studies have consistently found increased levels of interleukin-6
{[}IL-6; a known risk factor for depression @Howren2009{]} in response
to restricted sleep {[}@Reinhardt2016{]}, an effect that is resilient to
recovery sleep {[}@Simpson2016{]}. One mechanism by which this altered
inflammatory response affects cognitive and affective processing is via
the interoceptive system {[}@Critchley2013{]}. Afferent signals from
peripheral nerves that embed visceral organs communicate to the brain
what is happening physiologically in the body, including sensing
inflammation. Interoception interacts with many other cognitive
processes, such that our bodily feelings guide the way we behave
{[}@Critchley2017; @Rae2018{]}. Altogether, this suggests that
inflammation, via interoception, can drive how we feel and ultimately
how we act.

Neurally, the most consistent findings associated with inadequate sleep
are: (i) amygdala hyper-reactivity to aversive stimuli {[}@Yoo2007a;
@Motomura2014{]}; (ii) disconnect between frontal regions and the
amygdala, as well as the basal ganglia {[}@Yoo2007a; @Motomura2013;
@Prather2013; @Zhao2018{]}; (iii) altered structure and function in the
fronto-parietal network {[}@Cui2015; @Krause2017{]}. Furthermore, given
its pivotal role in both interoception and the salience-detection
network, the insular cortex is likely to be a key mediator of the
neurophysiological changes that result from chronic sleep restriction
{[}@Critchley2013; @Critchley2017; @Rae2018; @Ma2015{]}. However, few
studies that directly investigate interactions between work patterns,
inadequate sleep, and physiology, further assess neurobiological changes
in the same context.

This project will address the resulting gaps in the literature using a
combination of population neuroscience and epidemiological methods. We
will identify neural markers of occupational wellbeing in the UK Biobank
cohort: a population-based prospective study of \textasciitilde500,000
individuals, a subset of which completed an imaging follow-up, including
both task {[}i.e. emotional face perception; @Hariri2002a{]} and
resting-state functional Magnetic Resonance Imaging (fMRI)
{[}@Sudlow2015{]}. Following approval by the UK Biobank Access Committee
(Project ref. no.: 62188), brain imaging data (i.e.~task and resting
functional brain MRI data) will be obtained for 35,501 participants,
together with blood biochemistry assay results relevant to inflammation
(i.e.~C-reactive protein) and a curated selection of employment,
sociodemographic, lifestyle and health-related information collected
through questionnaires, verbal interviews and census data (e.g.~Townsend
Deprivation Scores). Details concerning the data analysis strategy can
be found under the \emph{Multifactorial analysis strategy} section
below.

\newpage

\hypertarget{hypotheses-and-predictions}{%
\subsection{Hypotheses and
predictions}\label{hypotheses-and-predictions}}

\textbf{\emph{H: Working patterns will be associated with specific
neural markers.}} Based on findings in the sleep literature
{[}@Yoo2007a; @Motomura2013; @Motomura2014; @Prather2013; @Zhao2018;
@Cui2015; @Krause2017; @Yeo2015; @DeHavas2012; @Shao2014; @Lei2015;
@Samann2010{]}, we predict that:

\begin{itemize}
\item
  Working longer hours will be associated with amygdala hyper-reactivity
  in response to emotional faces in the task fMRI.
\item
  Working longer hours will be associated with reduced functional
  connectivity between the prefrontal cortex and the amygdala in task
  and resting-state fMRI, an effect which may be mediated by the insular
  cortex.
\item
  Working longer hours will be associated with reduced functional
  connectivity within the salience-detection network, executive control,
  fronto-parietal, and default mode (DMN) networks in resting-state
  fMRI.
\item
  Working longer hours will be associated with attenuated
  anticorrelation between task-negative (i.e.~DMN) and task-positive
  regions (i.e.~executive control, fronto-parietal, and
  salience-detection networks) in resting-state fMRI.
\end{itemize}

\hspace{1cm}

\textbf{\emph{H: Occupational factors will affect physiological immune
function, in part, through altering sleep patterns.}} Given the known
neurophysiological consequences of sleep restriction {[}@Reinhardt2016;
@Simpson2016; @Meier-Ewert2004{]}, and the effect of working patterns on
sleep duration and quality {[}@Basner2014; @Basner2018;
@Marucci-Wellman2016, @VanderHulst2003{]}, we predict that:

\begin{itemize}
\tightlist
\item
  Working longer hours will be associated with higher concentrations of
  C-Reactive Protein (CRP) in the UK Biobank data.
\end{itemize}

\hspace{1cm}

\textbf{\emph{H: Work patterns will impact cognitive function and
workplace performance.}} Based on work time reduction interventions
{[}@Schiller2018; Schiller2017a; @Akerstedt2001{]}, we predict that:

\begin{itemize}
\tightlist
\item
  Working longer hours and other time-consuming occupational factors
  (such as commuting) will negatively correlate with cognitive measures
  in the UK Biobank.
\end{itemize}

\newpage

\hypertarget{multifactorial-analysis-strategy-original-26th-august-2020}{%
\subsection{Multifactorial analysis strategy (ORIGINAL; 26th August
2020)}\label{multifactorial-analysis-strategy-original-26th-august-2020}}

In order to determine how particular work patterns are associated with
neural markers, immune function, and cognition, we need to first
understand how numerous non-imaging/-physiological variables
inter-relate in our Biobank dataset. This includes employment,
sociodemographic, lifestyle, and health-related information. The
complete list of data fields requested from the Biobank for this project
can be found at \url{http://tiny.cc/UKBBMetaData_62188}
(N.B.:``Participants'' attribute refers to total participants in the
cohort with information in the selected ``Field'', not the number of
participants in the dataset linked to the current project).

It is plausible that, for example, number of hours worked (`\emph{Length
of working week for main job}') covaries in multiple, likely nonlinear,
ways, with other employment variables (such as commute time,
`\emph{distance between home and workplace}' \& `\emph{frequency of
travelling}'), and with numerous variables from other categories such as
sociodemographic, lifestyle, and health-related information. This means
that if we wish to test for a specific effect between hours worked and
neural function, for example, it may be challenging to conclude that a
neural characteristic is attributable to hours worked, without
accounting in some manner for these related variables. Even in the
imaging data alone, it is clear that numerous quality-related metrics
such as head motion, acquisition date and time, and scanner
configuration act as potential confounders {[}@Smith2020{]}.

Therefore, prior to analysing the imaging and physiological data, we
will undertake a comprehensive investigation of how employment variables
(e.g.~`\emph{Length of working week for main job}'; Biobank Field ID:
767) relate to sociodemographic, lifestyle, and health-related
information, in our dataset of \(n=35,501\) individuals. To do so, we
plan to use multivariate analysis techniques, incorporating machine
learning methods, potentially applying a Canonical Correlational
Analysis (CCA) approach.

A complete and full picture of these relationships is, although very
much worthy of scientific merit, not our primary and only goal, and also
a lengthier and more detailed process than is within our present remit.
We aim to characterise sufficiently the strongest inter-relations of
employment variables with sociodemographic, lifestyle, and
health-related information, and the nature of their association
(e.g.~linear, U-shape function). This will be a standalone piece of
work, but also form the preliminary stage to the investigation of brain,
body and behavioural markers of occupational wellbeing, to which the
above-stated hypotheses relate. For example, we may use variables that
show a strong, linear relation to number of hours worked as covariates
in mass univariate General Linear Models when analysing the Hariri task
fMRI. Other ways in which we utilise the picture of employment and other
non-imaging/-physiology variables in our analyses of the imaging and
physiology data will evolve as we proceed through the project, and will
be recounted in successive updates to this present preregistration
document.

\newpage

\hypertarget{multifactorial-analysis-strategy-updated-16th-october-2020}{%
\subsection{Multifactorial analysis strategy (UPDATED; 16th October
2020)}\label{multifactorial-analysis-strategy-updated-16th-october-2020}}

Our analysis will focus on the discovery of neuroimaging, immunological,
and cognitive markers of occupational factors that we predict have a
negative impact on health and wellbeing. In order to achieve this, we
will isolate subjects that present potentially deleterious employment
characteristics and match them with control cases (i.e.~subjects with
working patterns not considered to be deleterious) from the same cohort
(i.e.~UK Biobank) at a 1:1 ratio. Matching will take into account key
sociodemographic, lifestyle and environment, and health-related
characteristics, identified by variable selection methods described
below. We will then compare the deleterious employment group to the
control group on neuroimaging, immunological and cognitive markers, with
the control group matched to the case group on key sociodemographic,
lifestyle and environment, and health-related characteristics - ensuring
we can attribute observed group differences to occupational factors.

\hypertarget{occupational-factors}{%
\section{Occupational factors}\label{occupational-factors}}

In the first instance we will select a single, key, occupational factor
as our dependent variable/outcome of interest, i.e.~`\emph{Length of
working week for main job}' (Biobank Field ID: 767), and define our
population of interest as subjects who fall within the top 5\% (n=1,775)
of the distribution for this variable (n=35,501). This will identify a
group of \textasciitilde1,775 subjects who typically work long hours in
their main job (schematic in \emph{Figure 1}).

\textbf{xxx} \emph{Figure 1.} Defining our population of interest: we
will take the top 5\% of the sample distribution (n=35,501) for Biobank
Field ID 767, `\emph{Length of working week for main job}' to be our
population of interest.

Next, we will find a set of matched controls from the remaining 95\% of
the sample who work fewer hours. Our aim is to identify a group of
\textasciitilde1,775 subjects who work fewer hours than the case group,
but who are matched on key sociodemographic, lifestyle and environment,
and health-related characteristics (see \textbf{Identifying matching
variables by feature selection}).

Once we have completed our case-control analysis of neuroimaging,
immunological, and cognitive markers for `\emph{Length of working week
for main job}', we will seek to repeat the process for other Biobank
Field IDs of potentially deleterious occupational factors (such as
commute time or shift work). These subsequent analyses will likely
identify a different set of \textasciitilde1,775 cases and
\textasciitilde1,755 controls, although some individuals may be present
across different occupational factor analyses. \textbf{RAUL}: should we
list the potential additional future occ factors to be investigated
here? Or maybe we can leave for future PR? e.g. 1. \emph{Length of
working week for main job} 2. \emph{Commute time} 3. \emph{Shift work}
etc?

\hypertarget{identifying-matching-variables-by-feature-selection}{%
\section{Identifying matching variables by feature
selection}\label{identifying-matching-variables-by-feature-selection}}

In order to determine the variables that need to be included in the
matching process, we will compare the 5\% case group of interest to the
remaining 95\% of the sample on all Data Fields in our project 62188
dataset (see \url{http://tiny.cc/UKBBMetaData_62188}), excluding the
occupational factor of interest that defines the status of the 5\% case
group, and the neuroimaging, immunological, and cognitive markers of
interest. \textbf{RAUL}: are we going to have to do it repeatedly for
different MRI/CRP/non-physiology analysis?? I think we should keep the
same n=1,775 cases \& n=1,775 controls across MRI/CRP/etc. So we would
have to exclude all those variables we subsequently want to investigate
from the matching. We plan to use the following methods:

\begin{itemize}
\item
  \emph{Test statistics} (e.g.~t-test, chi-square) comparing the
  population of interest with the 95\% rest of the sample on all the
  possible matching variables. Variables with a Family Discovery Rate
  (FDR) adjusted p-value lower than a pre-defined number cut-off point
  (p\textless0.05) would be selected for matching. If this generates a
  very large number of matching variables with the consequence that it
  is impossible to identify at least \textasciitilde1,775 matched
  controls, we will take the variables with the most significant
  p-values to match on, progressively adding in further variables until
  we can no longer identify a minimum of \textasciitilde1,775 control
  individuals.
\item
  \emph{Random forests}, a classifier approach to identifying the most
  important features that characterise a dataset. This approach will be
  applied to estimate variable importance measures (e.g.~permutation
  importance), which will tell us how much each variable contributes to
  the characterisation of an individual as working long hours, and
  therefore, how important it is to match on this variable. We plan to
  use the \textbf{xxx} toolbox. All relevant variables, as determined by
  the selected importance scores, would be selected for matching. As
  with the \emph{test statistics} approach, if this generates a very
  large number of matching variables with the consequence that it is
  impossible to identify at least \textasciitilde1,775 matched controls,
  we will take the variables with the largest importance values to match
  on, progressively adding in further variables until we can no longer
  identify a minimum of \textasciitilde1,775 control individuals. We
  plan to run both the \emph{test statistics} and \emph{random forests}
  feature selection approaches to gain an overview of consistency
  between the two methods (i.e., if the same features appear as
  important to match cases and controls on, across the two approaches,
  this gives additional confidence).
\item
  \emph{Other means of achieving feature selection} may evolve as we
  proceed through the descriptive stages of the analysis, and will be
  recounted in successive updates to this present preregistration
  document.
\end{itemize}

\hypertarget{case-control-matching}{%
\section{Case-control matching}\label{case-control-matching}}

After determining the variables to include in the matching process,
appropriate distance measures will be determined for each variable
(e.g.~propensity scores), and a matching method implemented
(e.g.~nearest neighbour). \textbf{RAUL}: nice - well done for thinking
about this.

\newpage

\hypertarget{multistage-preregistration-plan}{%
\subsection{Multistage preregistration
plan}\label{multistage-preregistration-plan}}

We intend to successively update our preregistration documentation as we
proceed through each analysis step in this project. This document (dated
\#\#th October 2020) forms the second document in this series. As we
complete data analysis stages, we will be able to prescribe increasingly
specific hypotheses, and refine our original broad imaging and
physiological analysis approaches set out in earlier preregistration
documentation, to increasingly more precise plans. This will include an
evolution of expected statistical approaches as the multifactorial
complexities of our dataset reveal themselves in relation to our
specific questions about brain, body and behavioural markers of
occupational wellbeing.

We will seek to timestamp our data logs and analysis records, and
provide updated pregregistration documents in a public repository, such
that the evolutionary timeline is indicative of a commitment to
preregistration principles, (i.e.~to guard against questionable research
practices), while taking account of the multi-stage and complex nature
of this project.

\newpage

\end{document}
