% Options for packages loaded elsewhere
\PassOptionsToPackage{unicode}{hyperref}
\PassOptionsToPackage{hyphens}{url}
%
\documentclass[
  english,
  man]{apa6}
\usepackage{lmodern}
\usepackage{amssymb,amsmath}
\usepackage{ifxetex,ifluatex}
\ifnum 0\ifxetex 1\fi\ifluatex 1\fi=0 % if pdftex
  \usepackage[T1]{fontenc}
  \usepackage[utf8]{inputenc}
  \usepackage{textcomp} % provide euro and other symbols
\else % if luatex or xetex
  \usepackage{unicode-math}
  \defaultfontfeatures{Scale=MatchLowercase}
  \defaultfontfeatures[\rmfamily]{Ligatures=TeX,Scale=1}
\fi
% Use upquote if available, for straight quotes in verbatim environments
\IfFileExists{upquote.sty}{\usepackage{upquote}}{}
\IfFileExists{microtype.sty}{% use microtype if available
  \usepackage[]{microtype}
  \UseMicrotypeSet[protrusion]{basicmath} % disable protrusion for tt fonts
}{}
\makeatletter
\@ifundefined{KOMAClassName}{% if non-KOMA class
  \IfFileExists{parskip.sty}{%
    \usepackage{parskip}
  }{% else
    \setlength{\parindent}{0pt}
    \setlength{\parskip}{6pt plus 2pt minus 1pt}}
}{% if KOMA class
  \KOMAoptions{parskip=half}}
\makeatother
\usepackage{xcolor}
\IfFileExists{xurl.sty}{\usepackage{xurl}}{} % add URL line breaks if available
\IfFileExists{bookmark.sty}{\usepackage{bookmark}}{\usepackage{hyperref}}
\hypersetup{
  pdftitle={Characterizing the neural markers of occupational wellbeing},
  pdflang={en-EN},
  pdfkeywords={UK Biobank; Occupational factors; brain; fMRI},
  hidelinks,
  pdfcreator={LaTeX via pandoc}}
\urlstyle{same} % disable monospaced font for URLs
\usepackage{graphicx}
\makeatletter
\def\maxwidth{\ifdim\Gin@nat@width>\linewidth\linewidth\else\Gin@nat@width\fi}
\def\maxheight{\ifdim\Gin@nat@height>\textheight\textheight\else\Gin@nat@height\fi}
\makeatother
% Scale images if necessary, so that they will not overflow the page
% margins by default, and it is still possible to overwrite the defaults
% using explicit options in \includegraphics[width, height, ...]{}
\setkeys{Gin}{width=\maxwidth,height=\maxheight,keepaspectratio}
% Set default figure placement to htbp
\makeatletter
\def\fps@figure{htbp}
\makeatother
\setlength{\emergencystretch}{3em} % prevent overfull lines
\providecommand{\tightlist}{%
  \setlength{\itemsep}{0pt}\setlength{\parskip}{0pt}}
\setcounter{secnumdepth}{-\maxdimen} % remove section numbering
% Make \paragraph and \subparagraph free-standing
\ifx\paragraph\undefined\else
  \let\oldparagraph\paragraph
  \renewcommand{\paragraph}[1]{\oldparagraph{#1}\mbox{}}
\fi
\ifx\subparagraph\undefined\else
  \let\oldsubparagraph\subparagraph
  \renewcommand{\subparagraph}[1]{\oldsubparagraph{#1}\mbox{}}
\fi
% Manuscript styling
\usepackage{upgreek}
\captionsetup{font=singlespacing,justification=justified}

% Table formatting
\usepackage{longtable}
\usepackage{lscape}
% \usepackage[counterclockwise]{rotating}   % Landscape page setup for large tables
\usepackage{multirow}		% Table styling
\usepackage{tabularx}		% Control Column width
\usepackage[flushleft]{threeparttable}	% Allows for three part tables with a specified notes section
\usepackage{threeparttablex}            % Lets threeparttable work with longtable

% Create new environments so endfloat can handle them
% \newenvironment{ltable}
%   {\begin{landscape}\begin{center}\begin{threeparttable}}
%   {\end{threeparttable}\end{center}\end{landscape}}
\newenvironment{lltable}{\begin{landscape}\begin{center}\begin{ThreePartTable}}{\end{ThreePartTable}\end{center}\end{landscape}}

% Enables adjusting longtable caption width to table width
% Solution found at http://golatex.de/longtable-mit-caption-so-breit-wie-die-tabelle-t15767.html
\makeatletter
\newcommand\LastLTentrywidth{1em}
\newlength\longtablewidth
\setlength{\longtablewidth}{1in}
\newcommand{\getlongtablewidth}{\begingroup \ifcsname LT@\roman{LT@tables}\endcsname \global\longtablewidth=0pt \renewcommand{\LT@entry}[2]{\global\advance\longtablewidth by ##2\relax\gdef\LastLTentrywidth{##2}}\@nameuse{LT@\roman{LT@tables}} \fi \endgroup}

% \setlength{\parindent}{0.5in}
% \setlength{\parskip}{0pt plus 0pt minus 0pt}

% \usepackage{etoolbox}
\makeatletter
\patchcmd{\HyOrg@maketitle}
  {\section{\normalfont\normalsize\abstractname}}
  {\section*{\normalfont\normalsize\abstractname}}
  {}{\typeout{Failed to patch abstract.}}
\makeatother
\shorttitle{Neural markers of occupational wellbeing}
\author{Raul Ungureanu\textsuperscript{1, 2}\ \& Charlotte Rae\textsuperscript{2,3}}
\affiliation{
\vspace{0.5cm}
\textsuperscript{1} Sussex Neuroscience, School of Life Sciences, University of Sussex, Falmer, UK\\\textsuperscript{2} School of Psychology, University of Sussex, Falmer, UK\\\textsuperscript{3} Sackler Centre for Consciousness Science, University of Sussex, Falmer, UK}
\authornote{

Correspondence concerning this article should be addressed to Raul Ungureanu, Brighton, BN1 9QG, United Kingdom. E-mail: r.ungureanu@sussex.ac.uk}
\keywords{UK Biobank; Occupational factors; brain; fMRI\newline\indent Word count: X}
\DeclareDelayedFloatFlavor{ThreePartTable}{table}
\DeclareDelayedFloatFlavor{lltable}{table}
\DeclareDelayedFloatFlavor*{longtable}{table}
\makeatletter
\renewcommand{\efloat@iwrite}[1]{\immediate\expandafter\protected@write\csname efloat@post#1\endcsname{}}
\makeatother
\usepackage{csquotes}
\ifxetex
  % Load polyglossia as late as possible: uses bidi with RTL langages (e.g. Hebrew, Arabic)
  \usepackage{polyglossia}
  \setmainlanguage[]{english}
\else
  \usepackage[shorthands=off,main=english]{babel}
\fi
\newlength{\cslhangindent}
\setlength{\cslhangindent}{1.5em}
\newenvironment{cslreferences}%
  {\setlength{\parindent}{0pt}%
  \everypar{\setlength{\hangindent}{\cslhangindent}}\ignorespaces}%
  {\par}

\title{Characterizing the neural markers of occupational wellbeing}

\date{}

\abstract{
One or two sentences providing a \textbf{basic introduction} to the field, comprehensible to a scientist in any discipline.

Two to three sentences of \textbf{more detailed background}, comprehensible to scientists in related disciplines.

One sentence clearly stating the \textbf{general problem} being addressed by this particular study.

One sentence summarizing the main result (with the words ``\textbf{here we show}'' or their equivalent).

Two or three sentences explaining what the \textbf{main result} reveals in direct comparison to what was thought to be the case previously, or how the main result adds to previous knowledge.

One or two sentences to put the results into a more \textbf{general context}.

Two or three sentences to provide a \textbf{broader perspective}, readily comprehensible to a scientist in any discipline.
}

\begin{document}
\maketitle

\hypertarget{study-information}{%
\section{Study information}\label{study-information}}

\hypertarget{background}{%
\subsection{Background}\label{background}}

Work takes up a huge chunk of our adult lives: the average Briton works approximately 42 hours per week,\textsuperscript{{[}1{]}} with an additional \textasciitilde4.9 hours spent on commuting,\textsuperscript{{[}2{]}} and an estimate of \textasciitilde10.1 hours in unpaid overtime.\textsuperscript{{[}3{]}} These numbers have been growing in the past 30 years\textsuperscript{{[}1--3{]}} without benefits to productivity. Importantly, a growing body of evidence suggests a strong negative impact on our health and wellbeing. Long working hours are associated with higher risk of cardiovascular disease,\textsuperscript{{[}4{]}} higher incidence of depressive,\textsuperscript{{[}5{]}} and anxiety symptoms,\textsuperscript{{[}4{]}} deficient cognitive function,\textsuperscript{{[}6{]}} and adverse physiological changes.\textsuperscript{{[}7{]}} Moreover, interventional studies show that a reduction in working hours benefits both health and productivity.\textsuperscript{{[}8,9{]}} However, we do not yet understand the neurobiological implications of our modern, increasingly intense, working patterns. Three reasons motivate the need for such an understanding:

\begin{itemize}
\item
  The brain acts as an interface between the body and the environment, therefore, it is key for grasping the mechanism through which occupational factors are affecting our health and wellbeing.
\item
  Without it we cannot ascertain the true short-term impact of working patterns on our cognitive function and physiological health, let alone the long-term, potentially irreversible, effects on our mental health and wellbeing.
\item
  Scientific evidence is needed to inform public policy and industry standards surrounding healthy work patterns.
\end{itemize}

\newpage

\hypertarget{aim-and-objectives}{%
\subsection{Aim and objectives}\label{aim-and-objectives}}

This project aims to characterize the neurophysiological processes through which work patterns affect our health and wellbeing, with the following objectives:

\begin{itemize}
\tightlist
\item
  Identify occupational factors that have a meaningful impact on neuronal function and describe the mechanism of impact.
\item
  Assess how physiological inflammatory responses are altered by occupational factors.
\item
  Determine how the identified neuronal and inflammatory markers jointly affect our physical and mental health.
\end{itemize}

Progress against these objectives will help develop a holistic insight into why our wellbeing is affected by modern work patterns and other occupational factors.

\newpage

\hypertarget{rationale}{%
\subsection{Rationale}\label{rationale}}

Until recently the impact of working patterns on our neurophysiology has been overlooked, therefore it is difficult to formulate an investigative plan that directly builds on prior work. However, we have identified inadequate sleep as the principal means through which the influence of occupational factors on wellbeing is likely to manifest. First, relative to all other activities, work is the primary waking activity exchanged for sleep.\textsuperscript{{[}10{]}} Second, it is becoming increasingly common for workers to accumulate sleep debt throughout the working week and attempt to catch-up on the weekend, a countermeasure that has been shown to be ineffective in combating the deleterious effects of weekday sleep debt.\textsuperscript{{[}11--14{]}} Finally, working longer hours is associated with significantly reduced sleep duration and quality.\textsuperscript{{[}15{]}} Therefore, we will use the known neuronal and physiological mechanisms of sleep, and in particular sleep restriction, to help guide the incipient stage of our investigation.

A multitude of bodily systems react to and interact with sleep-loss, a key set being the body's inflammatory response, and in particular increased expression of proinflammatory cytokines. Sleep restriction studies consistently found increased levels of interleukin-6\textsuperscript{{[}16{]}} in response to restricted sleep,\textsuperscript{{[}13{]}} an effect that is resilient to recovery sleep.\textsuperscript{{[}12{]}} One mechanism in which this altered inflammatory response affects cognitive and affective processing is via the interoceptive system.\textsuperscript{{[}17{]}} Afferent signals from peripheral nerves that embed visceral organs communicate to the brain what is happening physiologically in the body, including sensing inflammation. Interoception interacts with many other cognitive processes, such that our bodily feelings determine the way we behave.\textsuperscript{{[}18,19{]}} Altogether, this suggests that inflammation, via interoception, can drive how we feel and ultimately how we act.

Neurally, the most consistent findings associated with inadequate sleep are: (i) amygdala hyper-reactivity to aversive stimuli;\textsuperscript{{[}20,21{]}} (ii) disconnect between frontal regions and the amygdala, as well as the basal ganglia {[}\textsuperscript{{[}20,22--24{]}}; (iii) altered structure and function in the fronto-parietal network.\textsuperscript{{[}25,26{]}} Furthermore, given its pivotal role in both interoception and the salience-detection network, the insular cortex is likely to be a key mediator of the neurophysiological changes that result from chronic sleep restriction.\textsuperscript{{[}17--19,27{]}} However, very few studies directly investigate interactions between work patterns, inadequate sleep, and physiology, with none of them further assessing neurobiological changes in the same context. This projects will address the resulting gaps in the literature using a combination of population and cognitive neuroscience, epidemiology and immunological methods.

\hypertarget{hypotheses-and-predictions}{%
\subsection{Hypotheses and predictions}\label{hypotheses-and-predictions}}

\begingroup
\setlength{\parindent}{-0.5in}
\setlength{\leftskip}{0.5in}

\hypertarget{refs}{}
\begin{cslreferences}
\leavevmode\hypertarget{ref-TUC1}{}%
1. TUC. (2019). British workers putting in longest hours in the EU, TUC analysis finds. Retrieved from \url{https://www.tuc.org.uk/news/british-workers-putting-longest-hours-eu-tuc-analysis-finds}

\leavevmode\hypertarget{ref-TUC2}{}%
2. TUC. (2019). Annual commuting time is up 21 hours compared to a decade ago, finds TUC. Retrieved from \url{https://www.tuc.org.uk/news/annual-commuting-time-21-hours-compared-decade-ago-finds-tuc}

\leavevmode\hypertarget{ref-TotallyMoney}{}%
3. TotallyMoney. (2019). Overtime Survey 2018 - how much overtime does the UK work? Retrieved from \url{https://www.totallymoney.com/overtime-survey/}

\leavevmode\hypertarget{ref-Bannai2014}{}%
4. Bannai, A., \& Tamakoshi, A. (2014). The association between long working hours and health: A systematic review of epidemiological evidence. \emph{Scandinavian Journal of Work, Environment \& Health}, \emph{40}(1), 5--18. \url{http://dx.doi.org/10.5271/sjweh.3388}

\leavevmode\hypertarget{ref-Kim2016}{}%
5. Kim, W., Park, E. C., Lee, T. H., \& Kim, T. H. (2016). Effect of working hours and precarious employment on depressive symptoms in South Korean employees: A longitudinal study. \emph{Occupational and Environmental Medicine}, \emph{73}(12), 816--822. \url{http://dx.doi.org/10.1136/oemed-2016-103553}

\leavevmode\hypertarget{ref-Kajitani2018}{}%
6. Kajitani, S., McKenzie, C., \& Sakata, K. (2018). Use It Too Much and Lose It? The Effect of Working Hours on Cognitive Ability. \emph{SSRN Electronic Journal}. \url{http://dx.doi.org/10.2139/ssrn.2737742}

\leavevmode\hypertarget{ref-VanderHulst2003}{}%
7. Hulst, M. van der. (2003). Long workhours and health. Finnish Institute of Occupational Health. \url{http://dx.doi.org/10.5271/sjweh.720}

\leavevmode\hypertarget{ref-Akerstedt2001}{}%
8. Akerstedt, T., Olsson, B., Ingre, M., Holmgren, M., \& Kecklund, G. (2001). \emph{A 6-hour working day--effects on health and well-being.} (Vol. 30, pp. 197--202). \url{http://dx.doi.org/10.11183/jhe1972.30.197}

\leavevmode\hypertarget{ref-Barck-Holst2017}{}%
9. Barck-Holst, P., Institutet, K., Åkerstedt, S. T., Hellgren, C., Nilsonne, Å., Åkerstedt, T., \ldots{} Hellgren, C. (2017). Reduced working hours and stress in the Swedish social services: A longitudinal study. \emph{International Social Work}, \emph{60}(4), 897--913. \url{http://dx.doi.org/10.1177/0020872815580045}

\leavevmode\hypertarget{ref-Basner2014}{}%
10. Basner, M., Spaeth, A. M., \& Dinges, D. F. (2014). Sociodemographic Characteristics and Waking Activities and their Role in the Timing and Duration of Sleep. \emph{Sleep}, \emph{37}(12), 1889--1906. \url{http://dx.doi.org/10.5665/sleep.4238}

\leavevmode\hypertarget{ref-Basner2018}{}%
11. Basner, M., Dinges, D. F., \& Basner, M. (2018). Ac c te d us cr ip t pt cr t. \emph{Sleep}, \emph{41}(4), zsy012. \url{http://dx.doi.org/10.1093/sleep/zsy012/4792945}

\leavevmode\hypertarget{ref-Simpson2016}{}%
12. Simpson, N. S., Diolombi, M., Scott-Sutherland, J., Yang, H., Bhatt, V., Gautam, S., \ldots{} Haack, M. (2016). Repeating patterns of sleep restriction and recovery: Do we get used to it? \emph{Brain, Behavior, and Immunity}, \emph{58}, 142--151. \url{http://dx.doi.org/10.1016/j.bbi.2016.06.001}

\leavevmode\hypertarget{ref-Reinhardt2016}{}%
13. Reinhardt, É. L., Fernandes, P. A. C. M. C. M., Markus, R. P., \& Fischer, F. M. (2016). Short sleep duration increases salivary IL-6 production. \emph{Chronobiology International}, \emph{33}(6), 780--782. \url{http://dx.doi.org/10.3109/07420528.2016.1167710}

\leavevmode\hypertarget{ref-Buxton2010}{}%
14. Buxton, O. M., Pavlova, M., Reid, E. W., Wang, W., Simonson, D. C., \& Adler, G. K. (2010). Sleep restriction for 1 week reduces insulin sensitivity in healthy men. \emph{Diabetes}, \emph{59}(9), 2126--2133. \url{http://dx.doi.org/10.2337/db09-0699}

\leavevmode\hypertarget{ref-Marucci-Wellman2016}{}%
15. Marucci-Wellman, H. R., Lombardi, D. A., \& Willetts, J. L. (2016). Chronobiology International The Journal of Biological and Medical Rhythm Research Working multiple jobs over a day or a week: Short-term effects on sleep duration. \url{http://dx.doi.org/10.3109/07420528.2016.1167717}

\leavevmode\hypertarget{ref-Howren2009}{}%
16. Howren, M. B., Lamkin, D. M., \& Suls, J. (2009). Associations of depression with c-reactive protein, IL-1, and IL-6: A meta-analysis. \emph{Psychosomatic Medicine}, \emph{71}(2), 171--186. \url{http://dx.doi.org/10.1097/PSY.0b013e3181907c1b}

\leavevmode\hypertarget{ref-Critchley2013}{}%
17. Critchley, H. D., \& Harrison, N. A. (2013, February). Visceral Influences on Brain and Behavior. Cell Press. \url{http://dx.doi.org/10.1016/j.neuron.2013.02.008}

\leavevmode\hypertarget{ref-Critchley2017}{}%
18. Critchley, H. D., \& Garfinkel, S. N. (2017). Interoception and emotion. \emph{Current Opinion in Psychology}, \emph{17}, 7--14. \url{http://dx.doi.org/10.1016/j.copsyc.2017.04.020}

\leavevmode\hypertarget{ref-Rae2018}{}%
19. Rae, C., Botan, V. E., Gould Van Praag, C. D., Herman, A. M., Nyyssönen, J. A. K., Watson, D. R., \ldots{} Critchley, H. D. (2018). Response inhibition on the stop signal task improves during cardiac contraction. \emph{Scientific Reports}, \emph{8}(1). \url{http://dx.doi.org/10.1038/s41598-018-27513-y}

\leavevmode\hypertarget{ref-Yoo2007a}{}%
20. Yoo, S. S., Gujar, N., Hu, P., Jolesz, F. A., \& Walker, M. P. (2007, October). The human emotional brain without sleep - a prefrontal amygdala disconnect. Elsevier. \url{http://dx.doi.org/10.1016/j.cub.2007.08.007}

\leavevmode\hypertarget{ref-Motomura2014}{}%
21. Motomura, Y., Kitamura, S., Oba, K., Terasawa, Y., Enomoto, M., Katayose, Y., \ldots{} Mishima, K. (2014). \emph{Sleepiness induced by sleep-debt enhanced amygdala activity for subliminal signals of fear}. \url{http://dx.doi.org/10.1186/1471-2202-15-97}

\leavevmode\hypertarget{ref-Motomura2013}{}%
22. Motomura, Y., Kitamura, S., Oba, K., Terasawa, Y., Enomoto, M., Katayose, Y., \ldots{} Mishima, K. (2013). Sleep Debt Elicits Negative Emotional Reaction through Diminished Amygdala-Anterior Cingulate Functional Connectivity. \emph{PLoS ONE}, \emph{8}(2), e56578. \url{http://dx.doi.org/10.1371/journal.pone.0056578}

\leavevmode\hypertarget{ref-Prather2013}{}%
23. Prather, A. A., Bogdan, R., \& Hariri, A. R. (2013). Impact of sleep quality on amygdala reactivity, negative affect, and perceived stress. \emph{Psychosomatic Medicine}, \emph{75}(4), 350--358. \url{http://dx.doi.org/10.1097/PSY.0b013e31828ef15b}

\leavevmode\hypertarget{ref-Zhao2018}{}%
24. Zhao, R., Zhang, X., Fei, N., Zhu, Y., Sun, J., Liu, P., \ldots{} Qin, W. (2018). Decreased cortical and subcortical response to inhibition control after sleep deprivation. \emph{Brain Imaging and Behavior}, \emph{13}(3), 638--650. \url{http://dx.doi.org/10.1007/s11682-018-9868-2}

\leavevmode\hypertarget{ref-Cui2015}{}%
25. Cui, J., Tkachenko, O., Gogel, H., Kipman, M., Preer, L. A., Weber, M., \ldots{} Killgore, W. D. S. (2015). Microstructure of frontoparietal connections predicts individual resistance to sleep deprivation. \emph{NeuroImage}, \emph{106}, 123--133. \url{http://dx.doi.org/10.1016/j.neuroimage.2014.11.035}

\leavevmode\hypertarget{ref-Krause2017}{}%
26. Krause, A. J., Simon, E. B., Mander, B. A., Greer, S. M., Saletin, J. M., Goldstein-Piekarski, A. N., \& Walker, M. P. (2017, July). The sleep-deprived human brain. Nature Publishing Group. \url{http://dx.doi.org/10.1038/nrn.2017.55}

\leavevmode\hypertarget{ref-Ma2015}{}%
27. Ma, N., Dinges, D. F., Basner, M., \& Rao, H. (2015). How Acute Total Sleep Loss Affects the Attending Brain: A Meta-Analysis of Neuroimaging Studies. \emph{Sleep}, \emph{38}(2), 233--240. \url{http://dx.doi.org/10.5665/sleep.4404}
\end{cslreferences}

\endgroup

\newpage

\hypertarget{design-plan}{%
\section{Design Plan}\label{design-plan}}

\hypertarget{study-type}{%
\subsection{Study type}\label{study-type}}

See OSF registration options.

\hypertarget{blinding}{%
\subsection{2. Blinding}\label{blinding}}

Blinding describes who is aware of the experimenta manipulations within a study.

\hypertarget{study-design}{%
\subsection{3. Study design}\label{study-design}}

\newpage

\end{document}
