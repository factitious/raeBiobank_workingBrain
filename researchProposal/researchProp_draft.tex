\PassOptionsToPackage{unicode=true}{hyperref} % options for packages loaded elsewhere
\PassOptionsToPackage{hyphens}{url}
%
\documentclass[]{article}
\usepackage{lmodern}
\usepackage{amssymb,amsmath}
\usepackage{ifxetex,ifluatex}
\usepackage{fixltx2e} % provides \textsubscript
\ifnum 0\ifxetex 1\fi\ifluatex 1\fi=0 % if pdftex
  \usepackage[T1]{fontenc}
  \usepackage[utf8]{inputenc}
  \usepackage{textcomp} % provides euro and other symbols
\else % if luatex or xelatex
  \usepackage{unicode-math}
  \defaultfontfeatures{Ligatures=TeX,Scale=MatchLowercase}
\fi
% use upquote if available, for straight quotes in verbatim environments
\IfFileExists{upquote.sty}{\usepackage{upquote}}{}
% use microtype if available
\IfFileExists{microtype.sty}{%
\usepackage[]{microtype}
\UseMicrotypeSet[protrusion]{basicmath} % disable protrusion for tt fonts
}{}
\IfFileExists{parskip.sty}{%
\usepackage{parskip}
}{% else
\setlength{\parindent}{0pt}
\setlength{\parskip}{6pt plus 2pt minus 1pt}
}
\usepackage{hyperref}
\hypersetup{
            pdftitle={Characterising the neurophysiological markers of occupational wellbeing},
            pdfauthor={Raul Ungureanu; Supervisor: Charlotte Rae},
            pdfborder={0 0 0},
            breaklinks=true}
\urlstyle{same}  % don't use monospace font for urls
\usepackage[margin=1in]{geometry}
\usepackage{longtable,booktabs}
% Fix footnotes in tables (requires footnote package)
\IfFileExists{footnote.sty}{\usepackage{footnote}\makesavenoteenv{longtable}}{}
\usepackage{graphicx,grffile}
\makeatletter
\def\maxwidth{\ifdim\Gin@nat@width>\linewidth\linewidth\else\Gin@nat@width\fi}
\def\maxheight{\ifdim\Gin@nat@height>\textheight\textheight\else\Gin@nat@height\fi}
\makeatother
% Scale images if necessary, so that they will not overflow the page
% margins by default, and it is still possible to overwrite the defaults
% using explicit options in \includegraphics[width, height, ...]{}
\setkeys{Gin}{width=\maxwidth,height=\maxheight,keepaspectratio}
\setlength{\emergencystretch}{3em}  % prevent overfull lines
\providecommand{\tightlist}{%
  \setlength{\itemsep}{0pt}\setlength{\parskip}{0pt}}
\setcounter{secnumdepth}{5}
% Redefines (sub)paragraphs to behave more like sections
\ifx\paragraph\undefined\else
\let\oldparagraph\paragraph
\renewcommand{\paragraph}[1]{\oldparagraph{#1}\mbox{}}
\fi
\ifx\subparagraph\undefined\else
\let\oldsubparagraph\subparagraph
\renewcommand{\subparagraph}[1]{\oldsubparagraph{#1}\mbox{}}
\fi

% set default figure placement to htbp
\makeatletter
\def\fps@figure{htbp}
\makeatother

\usepackage{fancyhdr}
\pagestyle{fancy}
\fancyhf{}
\fancyhead[L]{\bfseries Characterising the neurophysiological markers of occupational wellbeing | PhD Proposal}
\clearpage\maketitle
\thispagestyle{empty}
\fancyhead[R]{\thepage}
\renewcommand{\footrulewidth}{0.4pt}
\fancyfoot[R]{\copyright \  Raul Ungureanu, 2020}

\title{\textbf{Characterising the neurophysiological markers of occupational wellbeing}}
\author{Raul Ungureanu \and Supervisor: Charlotte Rae}
\date{}

\begin{document}
\maketitle
\begin{abstract}
The truth is, we spend most of our waking days working. Most people will spend a third of their adult lives at work\textsuperscript{{[}\protect\hyperlink{ref-WHO1995}{1}{]}}. The average Briton works approximately 42 hours per week\textsuperscript{{[}\protect\hyperlink{ref-TUC1}{2}{]}}, with an additional \textasciitilde{}4.9 hours spent on commuting\textsuperscript{{[}\protect\hyperlink{ref-TUC2}{3}{]}}. It should come as no surprise then that our work environment deeply affects our health and wellbeing. However, the neurophysiology associated with occupational factors, and the mechanism through which it influences wellbeing and mediates vulnerability to mental health symptoms, is largely unexplored. As a result, we are failing to recognise what puts certain individuals at risk, and others less so. This is a key research priority, as gaining an insight into these issues is needed to help guide preventative interventions, medical or ergonomic, that protect individuals in their workplace and the labour force as a whole.
\end{abstract}

\newpage

\hypertarget{outline}{%
\section{Outline}\label{outline}}

Research context (200w)

Aims (100w)

Rationale (500w)

Research outline (700w)

\begin{itemize}
\item
  Stage 1: UKBiobank.
\item
  Stage 2: MRI + Immunology data collection.
\item
  Stage 3: Analysis and write-up.
\end{itemize}

\hypertarget{research-context}{%
\section{Research context}\label{research-context}}

The truth is, we spend most of our waking days working. Most people will spend a third of their adult lives at work\textsuperscript{{[}\protect\hyperlink{ref-WHO1995}{1}{]}}. The average Briton works approximately 42 hours per week\textsuperscript{{[}\protect\hyperlink{ref-TUC1}{2}{]}}, with an additional \textasciitilde{}4.9 hours spent on commuting\textsuperscript{{[}\protect\hyperlink{ref-TUC2}{3}{]}}. It should come as no surprise then that our work environment deeply affects our health and wellbeing. However, the neurophysiology associated with occupational factors, and the mechanism through which it influences wellbeing and mediates vulnerability to mental health symptoms, is largely unexplored. As a result, we are failing to recognise what puts certain individuals at risk, and others less so. This is a key research priority, as gaining an insight into these issues is needed to help guide preventative interventions, medical or ergonomic, that protect individuals in their workplace and the labour force as a whole.

\hypertarget{aims-100w}{%
\section{Aims (100w)}\label{aims-100w}}

The aim of the present project is to fill this important knowledge gap and begin characterising the neural markers of occupational wellbeing.

\hypertarget{rationale-500w}{%
\section{Rationale (500w)}\label{rationale-500w}}

Three occupational factors have been identified and studied as primary risk factors for a variety of health and wellbeing outcomes: (i) (long) working hours; (ii) shift work; (iii) un- and under-employment\textsuperscript{{[}\protect\hyperlink{ref-Wong2019}{4}--\protect\hyperlink{ref-WhatWorksWellbeing2}{6}{]}}. There is ample evidence to show that long hours negatively impact physical health, both self-perceived {[}@{]}, and objectively measured (e.g.~higher risk of cardiovascular disease\textsuperscript{{[}\protect\hyperlink{ref-Bannai2014}{7}{]}}); mental health (e.g.~higher incidence of depressive\textsuperscript{{[}\protect\hyperlink{ref-Kim2016}{8}{]}}, and anxiety\textsuperscript{{[}\protect\hyperlink{ref-Bannai2014}{7}{]}} symptoms); cognitive function (e.g.~diminished performance on working memory and digit substitution tasks\textsuperscript{{[}\protect\hyperlink{ref-Kajitani2018}{9}{]}}) and health-promoting behaviours (e.g.~higher rate of tobacco and alcohol consumption\textsuperscript{{[}\protect\hyperlink{ref-Lallukka2008}{10}{]}}). Recent evidence suggests that even the established norm of \textasciitilde{}40h/week can be detrimental to cognitive ability and wellbeing\textsuperscript{{[}\protect\hyperlink{ref-Barck-Holst2017}{11}{]}}. Interventional studies on Swedish social workers found that reducing working hours (while retaining full salary) has positive, long-lasting effects on sleep, subjective stress measures, fatigue, negative emotion, and cognition {[}{[}\protect\hyperlink{ref-Barck-Holst2017}{11}{]}; Schiller2017a{]}. Together these findings suggest that long working hours are not only harmful to our wellbeing but could also be counterproductive. Moreover, it is not only the amount, but also when we work that can negatively affect our wellbeing. Shift work, usually defined as work outside the regular daytime work schedule of 9 - 5 {[}@{]}, has been linked with a wide range of negative health outcomes, physical (e.g.~increased risk of cancer, cardiovascular disease, diabetes, and asthma\textsuperscript{{[}\protect\hyperlink{ref-Maidstone2020}{12}{]}}), and psychological (e.g.~anxiety {[}@{]} and depression {[}@{]}). Shift work has also been shown to affect cognitive performance\textsuperscript{{[}\protect\hyperlink{ref-Vetter2012}{13}{]}}, with some evidence indicating that effects can be long-lived\textsuperscript{{[}\protect\hyperlink{ref-Weinmann2018}{15}{]}}. On the other hand, involuntary underemployment is one of the most damaging factors influencing individual wellbeing, with potentially permanent effects that extend far beyond what would be expected from reduced income, and similar in severity to bereavement\textsuperscript{{[}\protect\hyperlink{ref-WhatWorksWellbeing2}{6}{]}}.

\hypertarget{research-outline-700w}{%
\section{Research outline (700w)}\label{research-outline-700w}}

\newpage

\hypertarget{references}{%
\section*{References}\label{references}}
\addcontentsline{toc}{section}{References}

\hypertarget{refs}{}
\leavevmode\hypertarget{ref-WHO1995}{}%
1. WHO. (1995). Global strategy on occupational health for all: The way to health at work. World Health Organization. Retrieved from \url{https://www.who.int/occupational_health/globstrategy/en/}

\leavevmode\hypertarget{ref-TUC1}{}%
2. TUC. (2019). British workers putting in longest hours in the EU, TUC analysis finds. Retrieved from \url{https://www.tuc.org.uk/news/british-workers-putting-longest-hours-eu-tuc-analysis-finds}

\leavevmode\hypertarget{ref-TUC2}{}%
3. TUC. (2019). Annual commuting time is up 21 hours compared to a decade ago, finds TUC. Retrieved from \url{https://www.tuc.org.uk/news/annual-commuting-time-21-hours-compared-decade-ago-finds-tuc}

\leavevmode\hypertarget{ref-Wong2019}{}%
4. Wong, K., Chan, A. H., \& Ngan, S. C. (2019). The effect of long working hours and overtime on occupational health: A meta-analysis of evidence from 1998 to 2018. \emph{International Journal of Environmental Research and Public Health}, \emph{16}(12). \url{http://dx.doi.org/10.3390/ijerph16122102}

\leavevmode\hypertarget{ref-Caruso2014}{}%
5. Caruso, C. C. (2014). Negative impacts of shiftwork and long work hours. \emph{Rehabilitation Nursing}, \emph{39}(1), 16--25. \url{http://dx.doi.org/10.1002/rnj.107}

\leavevmode\hypertarget{ref-WhatWorksWellbeing2}{}%
6. Wellbeing, W. W. for. (2017). \emph{Unemployment, (Re)employment and Wellbeing}. Retrieved from \url{www.whatworkswellbeing.org}

\leavevmode\hypertarget{ref-Bannai2014}{}%
7. Bannai, A., \& Tamakoshi, A. (2014). The association between long working hours and health: A systematic review of epidemiological evidence. \emph{Scandinavian Journal of Work, Environment \& Health}, \emph{40}(1), 5--18. \url{http://dx.doi.org/10.5271/sjweh.3388}

\leavevmode\hypertarget{ref-Kim2016}{}%
8. Kim, W., Park, E. C., Lee, T. H., \& Kim, T. H. (2016). Effect of working hours and precarious employment on depressive symptoms in South Korean employees: A longitudinal study. \emph{Occupational and Environmental Medicine}, \emph{73}(12), 816--822. \url{http://dx.doi.org/10.1136/oemed-2016-103553}

\leavevmode\hypertarget{ref-Kajitani2018}{}%
9. Kajitani, S., McKenzie, C., \& Sakata, K. (2018). Use It Too Much and Lose It? The Effect of Working Hours on Cognitive Ability. \emph{SSRN Electronic Journal}. \url{http://dx.doi.org/10.2139/ssrn.2737742}

\leavevmode\hypertarget{ref-Lallukka2008}{}%
10. Lallukka, T., Lahelma, E., Rahkonen, O., Roos, E., Laaksonen, E., Martikainen, P., \ldots{} Kagamimori, S. (2008). Associations of job strain and working overtime with adverse health behaviors and obesity: Evidence from the Whitehall II Study, Helsinki Health Study, and the Japanese Civil Servants Study. \emph{Social Science \& Medicine}, \emph{66}(8), 1681--1698. \url{http://dx.doi.org/10.1016/j.socscimed.2007.12.027}

\leavevmode\hypertarget{ref-Barck-Holst2017}{}%
11. Barck-Holst, P., Nilsonne, Å., Åkerstedt, T., \& Hellgren, C. (2017). Reduced working hours and stress in the Swedish social services: A longitudinal study. \emph{International Social Work}, \emph{60}(4), 897--913. \url{http://dx.doi.org/10.1177/0020872815580045}

\leavevmode\hypertarget{ref-Maidstone2020}{}%
12. Maidstone, R., Turner, J., Vetter, C., Dashti, H. S., Saxena, R., Scheer, F. A., \ldots{} Durrington, H. J. (2020). Night Shift Work Increases the Risk of Asthma. \emph{medRxiv}, 2020.04.22.20074369. \url{http://dx.doi.org/10.1101/2020.04.22.20074369}

\leavevmode\hypertarget{ref-Vetter2012}{}%
13. Vetter, C., Juda, M., \& Roenneberg, T. (2012). The Influence of Internal Time, Time Awake, and Sleep Duration on Cognitive Performance in Shiftworkers. \emph{Chronobiology International}, \emph{29}(8), 1127--1138. \url{http://dx.doi.org/10.3109/07420528.2012.707999}

\leavevmode\hypertarget{ref-Baulk2009}{}%
14. Baulk, S. D., Fletcher, A., Kandelaars, K. J., Dawson, D., \& Roach, G. D. (2009). A field study of sleep and fatigue in a regular rotating 12-h shift system. \emph{Applied Ergonomics}, \emph{40}(4), 694--698. \url{http://dx.doi.org/10.1016/j.apergo.2008.06.003}

\leavevmode\hypertarget{ref-Weinmann2018}{}%
15. Weinmann, T., Vetter, C., Karch, S., Nowak, D., \& Radon, K. (2018). Shift work and cognitive impairment in later life - Results of a cross-sectional pilot study testing the feasibility of a large-scale epidemiologic investigation 11 Medical and Health Sciences 1117 Public Health and Health Services. \emph{BMC Public Health}, \emph{18}(1), 1256. \url{http://dx.doi.org/10.1186/s12889-018-6171-5}

\end{document}
